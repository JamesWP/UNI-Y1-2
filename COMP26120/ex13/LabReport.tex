\documentclass{article}

\title{COMP26120 Lab 13: Background}
\author{James Peach}

\begin{document}
\maketitle

% PART 1 %%%%%%%%%%%%%%%%%%%%%%%%%%%%%%%%%%%%%%%%%%%%%%%%%%%%%%%%%%%%%%%%%%%%%%

\section{The small-world hypothesis}
\label{sec:small world}
% Here give your statement of the small-world hypothesis and how you
% are going to test it.

The hypothesis predicts that on average the distance between any two people in
a social graph is 6 hops or 5 people between.

One might attempt to verify this by taking a graph and calculating the distance
between every node and then taking an average.


\section{Complexity Arguments}
\label{sec:complexity}
% Write down the complexity of Dijkstra's algorithm and of Floyd's algorithm.
% Explain why, for these graphs, Dijkstra's algorithm is more efficient.

Dijkstra's algo:

The complexity of the algorythm is dependant on two factors $V$ the number of 
vertexes or nodes and $E$ the number of edges or connections.

The complexity is as follows:

$$O(|E| + |V|log|V|)$$

Floyd–Warshall algo:

The complexity of this algorythm is given by:

$$O(|V^3|)$$

For the graphs we have there are a small number of edges per vertex in the
graph as the total connections possible for each vertex is $V-1$ however each
person only has $60-80$ friends. in Dijkstras algo the complexity is based on 
$|V|log|V|$ and this is much smaller than the $V^3$ in Floyd's

\section{Part 1 results}
\label{sec:part1}
% Give the results of part one experiments.


% PART 2 %%%%%%%%%%%%%%%%%%%%%%%%%%%%%%%%%%%%%%%%%%%%%%%%%%%%%%%%%%%%%%%%%%%%%%

\section{Part 2 complexity analysis}
\label{sec:complexity2}
% Give the complexity of the heuristic route finder.


\section{Part 2 results}
\label{sec:part2}
% Give the results of part two experiments.


\end{document}
