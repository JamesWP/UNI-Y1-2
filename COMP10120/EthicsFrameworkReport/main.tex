\documentclass[11pt]{article}
\usepackage[margin=0.5in]{geometry}
\author{James Peach}
\begin{document}
\title{Ethics Framework Report \textbf{IEEE Ethics Framework}}
\maketitle

\section{About the IEEE}
The \textbf{IEEE} (Institute of Electrical and Electronics Engineers) is an
association for electrical and electronics engineers. such organisations allow
for members to work to guidelines to produce and share dicoveries and
publications in the area. there are currently over three million documents in
the \textbf{IEEE Xplore� Digital Library}. The IEEE also has a code of ethics that
all members must abide by in any work they publish.

  
\section{The Code of Ethics}
The Code begins with a small declaration of the reassons the code exists and
then follows a list of the agreements.
\begin{quotation}
  ``We, the members of the IEEE, in recognition of the importance
  of our technologies in affecting the quality of life throughout the world, and in accepting 
  a personal obligation to our profession, its members and the
  communities we serve, do hereby commit ourselves to the highest ethical and
  professional conduct and agree''
\end{quotation}

\begin{quotation}
  ``to accept responsibility in making decisions consistent with
  the safety, health, and welfare of the public, and to disclose promptly factors
  that might endanger the public or the environment;''
\end{quotation}
The framework empasises the need to take responsibility for actions and
decisions this promotes responsible decision making and ensures that in the
event that somthing goes wrong or is put into question there is a person
with the responsibility for any decisions made.

\begin{quotation}
  ``to avoid real or perceived conflicts of interest whenever
  possible, and to disclose them to affected parties when they do exist;''
\end{quotation}
This point aims to reduce arguments between members of teams by making
sure members voice there oppinions with one another to help resolve issues before
they become a problem.

\begin{quotation}
  ``to be honest and realistic in stating claims or estimates based
  on available data;''
\end{quotation}
This point emphasises the importance that data or results must not be
faked in any way to back up claims or theories.

\begin{quotation}
  ``to reject bribery in all its forms;''
\end{quotation}

\begin{quotation}
  ``to improve the understanding of technology; its appropriate
  application, and potential consequences;''
\end{quotation}
This point helps to direct the research or work into new areas to help
improve our understanding in the subject area

\begin{quotation}
  ``to maintain and improve our technical competence and to
  undertake technological tasks for others only if qualified by training
  or experience, or after full disclosure of pertinent limitations;''
\end{quotation}
This point is an important one as working in a subject area that you are
not trained in can be dangerous and lead to mistakes if a full
understanding is not available. Some seamingly small area of work can be
critical in larger applications and if there is some misunderstanding then
it is the responsibility of the individual to gain understanding or to let
a superior know of the limitations.

\begin{quotation}
  ``to seek, accept, and offer honest criticism of technical work,
  to acknowledge and correct errors, and to credit properly the contributions of
  others;''
\end{quotation}
Collaberation is key to improving and spotting mistakes in technical works.
it offers a different point of view

\begin{quotation}
  ``to treat fairly all persons regardless of such factors as race, religion,
  gender, disability, age, or national origin;''
\end{quotation}
The gender and race of a person should not affect the content of the scientific
document. such factors will only slow down the study in the area and do not gain
any scientific value.

\begin{quotation}
  ``to avoid injuring others, their property, reputation, or employment by false
  or malicious action;''
\end{quotation}

\begin{quotation}
  ``to assist colleagues and co-workers in their professional development and to
  support them in following this code of ethics.''
\end{quotation}
A self supporting code of ehtics is nessasary to ensure that the rules are
followed and support is there for any qestions in proper ehtical practice.

\section{References}
\begin{itemize}
  \item {\textbf{IEEE Code of practice}
  (http://www.ieee.org/about/corporate/governance/p7-8.html)}
\end{itemize}
\end{document}
