\documentclass[12pt]{article}
\begin{document}

\author{James Peach}
\title{Quantum cryptography}


\maketitle

\section{Cryptography}
There are many applications of cryptography in technology from banking
to gaming. The ability to perform communication between two parties
without the ability for others to listen in on the conversation allows
some activities like sending money faster and opens up the possibility
for new things that were not possible before, like the ability to share
your work with others and for them to know that it was you that produced
the document.

Crypotography allows us to do all this and more! There are lots of
different types of cryptography:
  *  Hashing algorhtyms
  *  Symetric (secret) key crypto
  *  Antisymetric (public) key crypto
these all have very different uses and all do verry different things.

\subsection{Hashing algorhythims}
This family of crypto algorithyms are an effective method of turning
a input message of bytes and producing an output of a static length
so using MD5 as an example of a hashing algorythm on the binary conversion
of the string "Hello world!" produces the output 

"86FB269D190D2C85F6E0468CECA42A20"

this is a 16 byte hex string. now the usefullness of a hashing algorythm
is that if you change the input message the output or digest as it is known
is vastly different if we use a message of "Hello world?" instead the digest
is 

"48604754B9FED84B3FEEB84C5DC138C0" 

this property means that if i was sent
a very large message and its digest, i could very easily see if we both had
the same message by comparing the digest. even one mistake in the message
will produce a completley different digest.
now of course the number of different messages is infinite however given
the example of MD5 with its 128 bit digest you \textit{only} have ~ 3*10\^30
possible hashes. this means that there will be some messages that have the
same digest. this is a problem, consider you downloading a program allong
with the hash from a software company, you only want to run the program if
you can be sure that the program you have is exactly the one that was sent
because if you instead run a program and it turns out to be a virus your boss
will be very angry. if there was an easy way to create a program with the
same digest then it could be possible to fool you into running the attackers program
when two such messages both have the same digest this is called a hashing
collision. the effectivness of a hashing algotitym is the combination of
a small possibility of collisions and a quick calculation speed.

\subsection{Symetric key cryptography}
as the name suggests symetric key crypto is the process of encrypting a
message with a key or locking it. this prevents anyone without a key to
read the message. this allows the message to be passed through a third
party (in most cases the internet) to an intended recipient who has a
symetric (identical) key to the one you used to encrypt it. this type
of encryption allows large amounts of data to be sent securley however
the problem with this is that both parties must have agreed on a key before
needing to send a message. they must have a secure way to do this, they
could have a list of keys that they both use that was passed between
them without a third party having access to the key i.e. on a private
network. it should not be possible to find the key for an encrypted
message from simply examineing it no matter how long the data.

\subsection{Antisymetric key cryptography}
this is similar to the above symetric key cryptography, however the
interesting part about this is in the key pair. when a key is generated
from a random seed the algorythm creates two. one is chosen as the public
key and one as the private key. the public key is put on display by a
trusted thrid party the private key is never publicly visible and no one
but the creator has access. both keys can encrypt a message however
once a message has been encrypted only the other key can decrypt it.
this has two main implications:
  *  any file can be secureley sent to the (private) keyholder by
     encrypting the file with the public key. this ensures that
     only the keyholder can access the file
  *  the keyholder can hash a file and then encrypt the hash with the
     private key. this has the effect that anyone can verify the file
     to ensure that the creator has indeed had access to the original
these techniques allows applications to ensure that they can trust that
a user is indeed the creator of the message. for example the git server
verifys that the commit has not been tampered with by anyone on the network
and that the user was indeed the author of the commit as no one can create
a signed file without the private key.
the opposite is also usefull. your bank may publish their public key and
a digital signature with every web page or document. your computer can then
verify these documents were indeed created by the bank and no one between
has tampered with them.

\section{Quantum computing}
The advent of quantum computing means that the flaws discussed in the
processes above could be easily found and the gaurantees they give cannot
be trusted. Third parties could listen in on private conversations. 
files and pages on the internet cant be trusted or verified. this is possible
because with classical computation the effort required to break these gaurantees
is so impossibly large that the in the age of the universe not a sinlge one
could have been broken. however quantum computers offer techniques that
were before impossible classical computations can be made many orders 
of magnitude faster and things that were before impossible, possible

\subsection{Quantum cryptography}
Quantum cryptography is an effort to offer the above prinipals but instead
of offering security by mathematics, quantum cryptography bases the security
on quantum perculiarities that once proved will not offer any possible way
around them given any amount of computation available.

one such quantum perculiarity is the spin states of photons. The spin of
a photon is in one of 4 directions up-down, left-right top-left-bottom-right, 
top-right-bottom-left. the quantum nature of the photon means that you cannot mesure a photon
without disturbing it. and you can only mesure one set of directions at a time
if you mesure with a non-diagonal set of detectors then you will only get 
a correct answer if the spin is in one of the two states you chose to mesure
otherwise you get a random incorect answer and you cannot mesure a photon
twice.

lets consider two parties Alice and Bob, Alice wants to send a message to
bob without anyone knowing what was sent.
Alice will use standard symetric key encryption however to send the key
she will encode a string of photons with different spins and send them to
bob. in order for bob to decode the photons he will set up his detector 
in one of the two positions. he will incorrectly mesure half of the photons
however once he has made and recorded the results he sends a list of the
detectors that he used for each mesurement to alice. alice simply replies
with the mesurements for which bob used the correct detector. Bob then 
removes the incorect mesurements and this leaves a list of values that 
can be used as a key for standard encryption. because of the properties
of the quantum behavure of the particles any eavesdropping and mesurement
of the photons will lead to a message bieng unable to be decoded by bob.
this will lead to alice and bob knowing that thier message was comprimised.

now both parties have access to a collection of random bits, the same bits
in the same order. and we also have the physical knowlege that is is
impossible for anyone inbetween to get the code. when both parties have
these then this key can be used in a traditional method to encrypt any data.
the secret to having this form of encryption work is having a large key.
more traditional methods for key distribution rely on other forms of encryption
for example first encrypting the key with a public key and then sending 
the encrypted key publicly. the quantum method is more secure and dosen\'t
rely on a less secure method to transport the key.

\section{Conclusion}
Cryptography has been arround scince before the internet and has been
an ever evolving science with a basis on mahematical properties and 
computational complexity. however the quantum revolution has changed this
ever changing and breaking cycle of new algotithms into a more static and
proved method. of course at the moment only the key is transfered over the
new priniple however in the future the whole process can be secured in this
way and then we should be totaly safe from the prying eyes of others.
\end{document}
